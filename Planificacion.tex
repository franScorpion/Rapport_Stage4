%!TEX root =  tfg.tex
\chapter{Planificación}

\begin{quotation}[Novelist]{Ernest Hemingway (1899--1961)}
The good parts of a book may be only something a writer is lucky enough to overhear or it may be the wreck of his whole damn life -- and one is as good as the other.
\end{quotation}

\begin{abstract}
Resumen de lo que va a ocurrir en el capítulo. ¿Cuál es el objetivo que tenemos con este capítulo?
\end{abstract}

\section{Resumen temporal del proyecto}

\begin{table*}[htb]
	\centering
	\begin{coolTable}{ll}{2}
{Resumen del proyecto}
	\textbf{Fecha de inicio}&10/10/2014\\
	\textbf{Fecha de fin}&10/10/2014\\
	\textbf{Periodicidad de las revisiones}&3 semanas\\
	\textbf{Carga de trabajo semanal}&12 horas\\
	\textbf{Horas totales previstas}&225 horas\\ % entre 25-30 horas por crédito
	\textbf{Horas finales}&234 horas\\
	\end{coolTable}
	\caption{Tabla resumen de tiempos y planificación}
\end{table*}

\section{Planificación inicial}

Aquí un desglose de las iteraciones, comienzo y fin de cada una:

\begin{table*}[htb]
	\centering
	\begin{coolTable}{ll}{2}
{Resumen de iteraciones}
	\textbf{Iteración 1}&10/10/14 a 21/10/14\\
	\textbf{Iteración 2}&21/10/14 a 15/11/14\\
	\textbf{...}&dd/mm/aa a dd/mm/aa\\
	\end{coolTable}
	\caption{Planificación temporal de iteraciones}
\end{table*}

Explicar cómo se han decidido las fechas, interacción con fechas importantes y situaciones personales.

\textbf{ESTE CAPÍTULO DEBE ESCRIBIRSE AL COMIENZO DEL PROYECTO}

\section{Informe de tiempos del proyecto}

Lo mismo que el anterior pero con datos reales. Ver Tabla \ref{tab:InformeTiempos}.

\begin{table*}[htb]
	\centering
	\begin{coolTable}{ll}{2}
{Resumen de iteraciones}
	\textbf{Iteración 1}&10/10/14 a 21/10/14\\
	\textbf{Iteración 2}&21/10/14 a 15/11/14\\
	\textbf{...}&dd/mm/aa a dd/mm/aa\\
	\end{coolTable}
	\caption{Planificación temporal de iteraciones\label{tab:InformeTiempos}}
\end{table*}

Justificar los retrasos de forma detallada aquí para cada una de las iteraciones. Explicar las razones.