%!TEX root =  tfg.tex
\chapter{Introduction}

\begin{quotation}[Novelist]{Ernest Hemingway (1899--1961)}
The good parts of a book may be only something a writer is lucky enough to overhear or it may be the wreck of his whole damn life -- and one is as good as the other.
\end{quotation}

\begin{abstract}
Resumen de lo que va a ocurrir en el capítulo. ¿Cuál es el objetivo que tenemos con este capítulo?
\end{abstract}

\section{Contexte}

Le 14 Avril 2016, une nouvelle règlementation sur la protection des données à caractère personnel est entrée en vigueur ; laissant une marge de deux années à toute entreprise manipulant des données à caractère personnel d’individus ressortissant de l’UE le soin de s’y conformer. Mazars Cameroun, branche de Mazars Group, a décidé de se conformer à cette réglementation entrée en vigueur le 25 Mai 2018. Cette mise en conformité a pour principal objectif d’accroître à la fois la sécurité des données à caractère personnel manipulées par l’entreprise mais aussi, de responsabiliser les acteurs de ce traitement. Ainsi donc, la classification des informations manipulées est incontournable pour la mise en conformité de l’entreprise. 

\section{Problématique}

Classifier une donnée c’est lui attribuer une mention qui permet de caractériser la valeur et l’importance stratégique de cette donnée détenue par une entreprise et, conséquemment le niveau de protection à lui accorder.

Comment mettre en place une solution de classification efficace qui va pouvoir garantir la sécurité des informations manipulées au sein de Mazars Cameroun ? 

Quelles sont les différentes procédures à mettre sur pied afin d’assurer la mise en conformité de Mazars Cameroun non seulement au RGPD, mais aussi aux lois en vigueur ?


\section{Objectifs et Motivations}

Les objectifs visés par la classification des données manipulées au sein de Mazars Cameroun sont les suivants :
\begin{itemize}
  \item Accroître la protection des données à caractère personnel manipulées au sein de la structure ;
  \item Responsabiliser les acteurs du traitement de ces données ;
  \item Renforcer les droits des personnes ;
  \item Crédibiliser la régulation.
\end{itemize}

Comme motivations à ce projet de mise en conformité, la première est les sanctions prévues par le RGPD en cas de non-conformité ; ces sanctions peuvent aller jusqu’à une amende de 20 Millions d’Euros. En plus, la mise en conformité au RGPD est un critère non négligeable requis par certains clients en ce qui concerne l’acquisition de nouveaux marchés. Par ailleurs, la mise en conformité RGPD est un service que pourrait proposer la structure mais il faudrait premièrement qu’elle-même y soit conforme. 