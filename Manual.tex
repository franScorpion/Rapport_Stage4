%!TEX root =  tfg.tex
\chapter{Conception de la solution et mise en oeuvre}

\begin{quotation}[Novelist]{Ernest Hemingway (1899--1961)}
The good parts of a book may be only something a writer is lucky enough to overhear or it may be the wreck of his whole damn life -- and one is as good as the other.
\end{quotation}

\begin{abstract}
Resumen de lo que va a ocurrir en el capítulo. ¿Cuál es el objetivo que tenemos con este capítulo?
\end{abstract}

\section{Réalisation de la cartographie des données}
\subsection{Problématique}
Le problème ici est de pouvoir tracer le parcours d’une donnée depuis son entrée dans l’entreprise jusqu’à sa sortie ou sa destruction. Comment est créée l’information ? Où est-elle stockée ? Combien de temps est-elle stockée ? Quel est son procédé de destruction ? Qui a accès à cette information ? Comment cette information est-elle gardée en sécurité ?

\subsection{Cartographie}
Pour se faire l’on doit recenser un certain nombre d’éléments lors de la cartographie des processus :
\begin{itemize}
    \item[-] \textbf{Le nom de l’information : }identifiant unique de l’information au sein de l’entreprise ;
    \item[-] \textbf{Le propriétaire de l’information : }responsable de la gestion de l’information ;
    \item[-] \textbf{Le format de l’information : }format physique ou numérique principalement ;
    \item[-] \textbf{Les responsabilités liées à l’information : }Propriété, entités qui y accèdent et droits d’accès ;
    \item[-] \textbf{L’impact que l’information peut avoir sur l’entreprise : }conformité, opérationnel, réputation, financier ;
    \item[-] \textbf{Ses besoins de sécurité : }en termes de confidentialité, d’intégrité et de disponibilité ;
    \item[-] \textbf{Canaux de diffusion : }SIs sollicités, systèmes et réseaux utilisés pour le transfert de l’information ;
    \item[-] \textbf{Les règles de marquage applicable :}lois et règles applicables sur l’information ;
    \item[-] \textbf{Le cycle de vie de la donnée : }temps de conservation jusqu’à étape du processus
    \item[-] \textbf{Le support de stockage : }Cloud, Bases de données, serveurs…
    \item[-] \textbf{La conservation : }temps de conservation total de l’information ;
    \item[-] \textbf{Le procédé de destruction de l’information ;}
    \item[-] \textbf{Les mesures de sécurité appliquées dans l’entreprise.}
\end{itemize}
Un extrait de cette cartographie est fournie dans un document annexe nommé « Cartographie des traitements ».

Il est important de faire ressortir les échelles de classification suivant les 3 principes de la sécurité : confidentialité, intégrité, disponibilité. Le tableau ci-dessous fournit les échelles de classification que nous avons retenue :

\begin{center}
    \begin{tabular}{|c|c|c|c}
        \hline
        \multicolumn{4}{|c|}{Echelle de confidentialité} \\
         Confidentialité & Description de l'expression du besoin & Niveau d'impact redouté & Exemple  \\
         Confidentiel & Information ayant un impact significatif sur l’entreprise si elle était amenée à être divulguée en dehors de personnes identifiées & 4 & Préparation d’une nouvelle offre de service \\
         A usage restreint & Information dont l’accès est restreint au personnel (interne) impliqué & 3 & Dossier de mission \\
         A usage interne & Information dont la divulgation en dehors de l’entreprise peut nuire à celle ci & 2 & Organigramme  l’entreprise \\
         Public & Information qui peut être rendue publique sans impact pour l’entreprise ou l’entité associée & 1 & Adresse du cabinet \\
         \hline
        \multicolumn{4}{|c|}{Echelle d'Intégrité} \\
        Intégrité & Description de l'expression du besoin & Niveau d'impact redouté & Exemple \\
        Elevé & Information dont l’altération provoquerait un impact élevé sur l’entreprise & 3 & Données financières de l'entreprise\\
        Moyen & Information dont l’altération aura un impact important sur l’entreprise & 2 & Contenu du site internet de Mazars Cameroun \\
        Faible & Toute altération de l’information aura un impact faible sur l’entreprise & 1 & Livret de présentation Mazars Cameroun \\
        \hline
        \multicolumn{4}{|c|}{Echelle de disponibilité}\\
        Disponibilité & Description de l'expression du besoin & Niveau d'impact redouté & Exemple \\
        Elevé & Indisponibilité non tolérée. Toute indisponibilité aura un impact élevé sur l’entreprise & 3 & Serveur de messagerie \\
        Moyen & Toute indisponibilité aura un impact important sur l’entreprise & 2 & GESTRAS \\
        Faible & L’indisponibilité a un impact faible pour l’entreprise & 1 & Site Web \\
        \hline 
    \end{tabular}
\end{center}
\section{Modélisation des processus métier}
Le tableau ci-dessous présente un récapitulatif des processus métiers que l’on a pu recenser suivant le corps de métier au sein de la structure.
\begin{center}
    \begin{tabular}{|c|c|}
        \hline
         Corps de métier & Nom du processus  \\
         \multirow{8}{4em}{Ressources Humaines} & Recrutement \\
         &Mise en Service \\
         & Envoi en formation \\
         & Traitement de la paie \\
         & Gestion des évaluations \\
         & Départ en congés \\
         & Renvoi/Démission \\
         & Archivage \\
         \hline
         \multirow{4}{2em}{Tax/Legal} & Ouverture d'une mission \\
         & Organisation des documents collectés pour une mission \\
         & Clôture d’une mission \\
         & Archivage \\
         \hline
         \multirow{6}{2em}{Audit} & Création d'une mission \\
         & Traitement du dossier d’audit \\
         & Gestion du dossier de mission \\
         & Gestion du dossier permanent \\
         & Gestion de la fin d’une mission \\
         & Archivage \\
         \hline
         \multirow{3}{2em}{Equipe IT} & Stockage \\
         & Sauvegarde \\
         & Archivage \\
         \hline
         \multirow{2}{2em}{Administration et Finances}
         & Traitement des informations bancaires et financières \\
         \hline
         Business Development & Traitement des données utilisées pour les appels d’offres, proposition de service \\
    \end{tabular}
\end{center}
La description de chaque processus est fournie dans l’annexe « Cartographie des traitements ».
\textit{Département RH}
Le département RH possède déjà ces propres processus, l'on a donc proposé des améliorations aux processus existants.

\textit{Département Audit}
\textit{Département Tax/Legal}
\textit{Administration et Finances}
\section{Rédaction des procédures}
A la suite de la classification effectuée, sur la base des besoins en sécurité des biens essentiels et sur les mesures de sécurité à mettre en place, nous avons rédigé des procédures renseignées dans le « Manuel de procédures » de l’entreprise.

\section{Mise en oeuvre des procédures}
\textbf{Plan de déploiement}
\begin{tabular}{c|c}
    Responsables des mesures de sécurité identifiées & Le délégué à la protection des données(DPO)  \\
     Echéances de déploiement & Aucune échéance ne peut être fournie car toutes les politiques et procédures n’ont pas encore été mises en place \\
     Ressources nécessaires & - \\
     Propriétaires de l'information & Délégué à la protection des données \\
     Responsable du suivi et du maintien de la classification & Délégué à la protection des données \\
     Parties prenantes impliquées dans la revue de la classification & \begin{itemize}
         \item[-] Délégué à la protection des données (DPO)
         \item[-] Tout membre du personnel ayant été sensibilisé sur la classification des données
     \end{itemize}\\
\end{tabular}