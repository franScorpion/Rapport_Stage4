%!TEX root =  tfg.tex
\chapter{Conclusiones}

\begin{quotation}[Novelist]{Ernest Hemingway (1899--1961)}
The good parts of a book may be only something a writer is lucky enough to overhear or it may be the wreck of his whole damn life -- and one is as good as the other.
\end{quotation}

\begin{abstract}
Resumen de lo que va a ocurrir en el capítulo. ¿Cuál es el objetivo que tenemos con este capítulo?
\end{abstract}

\section{Rappel du problème}

Le problème qui nous a été posé avait deux volets ; premièrement mettre en place une solution de classification des données manipulées par Mazars Cameroun. Suite à cette classification, rédiger des politiques et procédures qui permettront d’assurer la conformité de Mazars Cameroun au RGPD ainsi qu’aux autres lois sur la protection des données.

\subsection{Bilan}
Au terme de notre stage, nous avons proposé une classification des différentes informations que l’on a pu recenser, ce suivant l’échelle de classification fournie plus haut et suivant l’impact que pourrait avoir cette information si son besoin en sécurité venait à être modifié. Cette classification a été établie dans un document appelé « Cartographie des traitements » qui associe à chaque information un marquage approprié (confidentiel, public, etc.). En plus de ce document de classification nous avons fourni le « registre de traitements ». Notons que l’article 30 du GDPR impose la tenue d’un registre de traitement qui contient un certain nombre d’informations sur les traitements impliquant les données personnelles. 

En plus, nous avons renseigné dans un « Manuel de procédures » un certain nombre de tâches à exécuter afin de pouvoir réaliser les processus métiers au sein de l’entreprise et ce en respect de la règlementation en vigueur. 

\subsection{Acquis personnels et difficultés rencontrées}
Ce stage nous a permis:
\begin{itemize}
    \item[-] D'améliorer notre sens de la communication à travers les interviews menées;
    \item[-] De découvrir le RGPD et d’augmenter notre niveau de connaissances en matière de protection des données (notamment dans la mise en place de mesures de sécurité appropriées) ;
    \item[-] D’acquérir des connaissances sur le fonctionnement global d’un cabinet d’Audit, ainsi que les responsabilités liées au travail en cabinet.
\end{itemize}
Difficultés rencontrées:
\begin{itemize}
    \item[-] La sensibilisation du personnel n’a pas été effectuée : ainsi, pendant les interviews les éléments collectés n’étaient pas toujours pertinents car les intervenants n’avaient pas été préparés à notre venue;
    \item[-] Les interlocuteurs n’avaient n’étaient pas toujours qualifiés ou bien informés pour répondre à nos questions, ceci due à l’indisponibilité des chefs de départements;
    \item[-] -	Mis à part le département des Ressources Humaines, aucun autre département n’avaient des procédures rédigées et validées ; cela est la cause de l’importance du temps que nous avons mis à recenser les éléments qu’il fallait pour les procédures.
\end{itemize}

\subsection{Perspectives}
Dans un souci d’amélioration de ce qui a été fait jusqu’ici, il conviendrait de :
\begin{itemize}
    \item[-]Renseigner dans le manuel de procédures toutes les procédures recensées au sein de l’entreprise car ce n’est qu’une partie des procédures qui a été rédigée ;
    \item[-]Migrer vers une solution automatique de classification des documents, celle-ci aura pour principal objectif de marquer les documents (Word, Powerpoint, etc.) à chaque niveau de leur cycle de vie ;
    \item[-] Pouvoir identifier le lieu de stockage de chaque donnée au sein de l’entreprise, et tenir en compte le fait qu’elle ait pu être dupliquée ;
    \item[-] Mettre en place des politiques de suppression des données conformément au RGPD ;
    \item[-] Automatiser un certain nombre de processus afin de faciliter la traçabilité des données manipulées et afin de se rassurer que les procédures sont respectées.
\end{itemize}

