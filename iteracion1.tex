%!TEX root =  tfg.tex
\chapter{Costes}

\begin{quotation}[Novelist]{Ernest Hemingway (1899--1961)}
The good parts of a book may be only something a writer is lucky enough to overhear or it may be the wreck of his whole damn life -- and one is as good as the other.
\end{quotation}

\begin{abstract}
Resumen de lo que va a ocurrir en el capítulo. ¿Cuál es el objetivo que tenemos con este capítulo?
\end{abstract}

\section{Características a desarrollar}

\begin{enumerate}
\item Funcionalidad A. Ver Tabla \ref{tab:valorAportado1}.
\item Funcionalidad B.
\end{enumerate}

\begin{table*}[htb]
	\centering
	\begin{coolTable}{p{4cm}p{\textwidth-4.5cm}}{2}
{Análisis de valor aportado 0001}
\textbf{Propuesta}&Trabajo que pretende analizarse y justificarse\\
\midrule
\textbf{Valor}&Qué valor aporta al proyecto o al usuario final.\\
\textbf{Coste}&Qué costes en términos de esfuerzo, adquisiciones y limitaciones tiene la propuesta\\
\textbf{Opciones}&Qué otras opciones se tienen que aporten un valor similar? ¿Es realmente un valor relevante para el proyecto/cliente\\
\textbf{Riesgos}&Qué riesgos pueden surgir a la hora de desarrollar esta propuesta.\\
\textbf{Deuda técnica}&Posibles deudas técnicas que se asumen con el desarrollo de esta propuesta.\\
	\end{coolTable}
	\caption{Análisis de valor aportado 0001\label{tab:valorAportado1}}
\end{table*}

\section{Diseño}
Aquí una discusión de cómo va a afectar todo al diseño

Debe insertarse un diagrama UML de diseño con los cambios y hacer referencia en el texto así Fig. \ref{fig:diseno01}.

\begin{figure}[htbp]
\begin{center}
\missingfigure{Aquí el modelo de diseño en formato vectorial preferentemente (pdf)}
% Incluir la figura quitando el comentario a la fila de abajo.
% \includegraphics[width=\textwidth]{myfile.pdf}
\caption{Diagrama UML de diseño para la iteración 1}
\label{fig:diseno01}
\end{center}
\end{figure}

Un memorando técnico por cada decisión de diseño.

\begin{table*}[htb]
	\centering
	\begin{coolTable}{p{4cm}p{\textwidth-4.5cm}}{2}
{Memorando técnico 0001}
\textbf{Asunto}&¿Cuál es el problema?\\
\textbf{Resumen}&¿Cuál es la solución propuesta?\\
\midrule
\textbf{Factores causantes}&Descripción pormenorizada del problema\\
\textbf{Solución}&Descripción pormenorizada de la solución propuesta\\
\textbf{Motivación}&¿Por qué propone esta solución?\\
\textbf{Cuestiones abiertas}& Factores a tener en cuenta en la solución cuya dimensión se reconoce.\\
\textbf{Alternativas}&Otras soluciones consideradas y la razón por la que se excluyeron.\\
	\end{coolTable}
	\caption{Memorando técnico 0001}
\end{table*}


\section{Implementación}

Un memorando técnico por cada decisión de implementación y refactorización que afecte al diseño.

\begin{asigResponsabilidad}{0001}{Prueba}
{[return\_type] method\_name1 (param1:type1, ...)}
\pasoPseudo{1. Paso 1.}
\pasoPseudo{2. Paso 2.}
\cabeceraMetodosBajoNivel
\pasoCodigo{1}{ClassName}{[return\_type] method\_name1 (param1:type1, ...)}{001}{SI}
\diagramaColaboracion{figures/colDiagram.png}
\end{asigResponsabilidad}

\begin{asigResponsabilidad}{alvotermar02}{Grubber}
{[return\_type] grubber (param1:type1, ...)}
\pasoPseudo{1. Lanzar 2 dados}
\pasoPseudo{2. Compara resultado de los dados con kicking del open-side}
\pasoPseudo{2.1. Si valor dados es menor o igual a kicking, avanza 10m}
\pasoPseudo{3.1. Si no hay defensa y el golpeo es exitoso, el pateador retiene la posesi\'on del bal\'on}
\pasoPseudo{3.2. Si hay defensa y el golpe es exitoso, el atacante tira un dado y suma su valor al de speed y strength y el defensor lanza 2 dados y lo suma al valor de speed y strength de su jugador, el vencedor ser\'a aquel que tenga m\'as puntos, si es igual, la posesi\'on es del defensor}
\pasoPseudo{4.1. Si no es exitoso y hay defensa el bal\'on pasa a posesi\'on del defensor}
\pasoPseudo{4.2. Si no es exitoso y no hay defensa de lanza un line-out}
\cabeceraMetodosBajoNivel
\pasoCodigo{1}{Dice}{[Integer] throwDice ()}{001}{SI}
\pasoCodigo{2}{ClassName}{[Int] compareKickingToDice (kicking:Integer, dice: Integer)}{001}{SI}
\pasoCodigo{2.1}{ClassName}{[Integer] setLine (line:Integer)}{001}{SI}
\pasoCodigo{4.2}{ClassName}{[Integer] lineOut ()}{001}{SI}
%\diagramaColaboracion{colDiagram.png}
\end{asigResponsabilidad}

\section{Pruebas}

Descripción de las pruebas realizadas al software

\section{Despliegue}

Breve resumen de cómo se han desplegado los cambios en el sistema de producción.